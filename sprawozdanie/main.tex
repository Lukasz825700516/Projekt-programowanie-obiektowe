\documentclass{article}

\usepackage[utf8]{inputenc}
\usepackage{polski}
\usepackage[polish]{babel}

\usepackage[normalem]{ulem}
\usepackage{adjustbox}
\usepackage{array,booktabs}
\usepackage{amssymb}

\makeatletter
\def\UL@putbox{\ifx\UL@start\@empty \else % not inner
  \vrule\@width\z@ \LA@penalty\@M
  {\UL@skip\wd\UL@box \UL@leaders \kern-\UL@skip}%
    \phantom{\box\UL@box}%
  \fi}
\makeatother

\begin{document}
\section{Projekt}
Projekt z programowania obiektowego, symulacji agentowej
\uline{AAAA AAAA AAAA AAAA AAAA AAAA AAAA}
przy użyciu języka programowania
\underline{\textbf{java}}.

\subsection{Grupa projektowa}
\begin{center}
	\begin{tabular}{cccc}
		Imię i Nazwisko & Rola & Numer indeksu & Termin labolatorii \\
		\hline Łukasz Mędrek & lider & 263910 & Środa 11:15 \\
		Jakub Kalecki & & 263962 & Środa 11:15 \\
	\end{tabular}
\end{center}

\subsection{Cel projektu}
Celem projektu jest zastosowanie technik projektowania obiektowego w  
w celu utworzenia modelu wybranego wycinka rzeczywistości i obserwacji
występujących w nim interakcji.

Projekt powinien mieć możliwość wprowadzenia danych początkowych, jak 
i zapisywania danych przeprowadzonej symulacji.

\subsection{Proponowane tematy}
\begin{adjustbox}{center}
	\begin{tabular}[h]{
			>{\centering\arraybackslash}c
			p{0.45\textwidth}
			p{0.45\textwidth}
			>{\centering\arraybackslash}c
		}
			Symulacja & \centering{Opis} & \centering{Cel} & Wybór \\
			% \hline Prosty ekosystem & {
				% Ekosystem składa się z różnych drapieżników, roślinożerców
				% i roślin, drapieżniki polują na roślinożerców, a roślinożercy
				% wędrują po planszy w poszukiwaniu roślin.
			% } & { 
				% Obserwacja cech każdej z grup w poszukiwaniu zależności
				% modyfikujących ich wielkość populacji.
			% }&  \\
			\hline \centering{Bitwa samolotowa} & { 
				Samoloty dwóch lub więcej stron prowadzą ze sobą
				bitwę w przestworzach, i mają za zadanie zestrzelić
				wrogie sobie jednostki.
			} & { 
				Obserwacja cech każdej z stron konfliktu w poszukiwaniu
				zależności modyfikujących ich szybkość eliminacji wrogich
				samolotów jak i końcową ilość jednostek.
			}& \checkmark \\
			\hline \centering{Statki} & { 
				Wersja gry w statki, w której każdy ze statków podwodnych może stwierdzić
				czy wokół niego znajduje się obca jednostka za pomocą radaru z daną dokładnością,
				wtedy rozpoczyna on ostrzał obszaru w którym wykrył wroga,
				próbując go wyeliminować. Statki mogą się poruszać, ale nie mogą się
				ze sobą zderzać.
			} & { 
				Obserwacja cech każdej z stron konfliktu w poszukiwaniu
				zależności modyfikujących ich szybkość eliminacji wrogich
				samolotów jak i końcową ilość jednostek.
			}&  \\
		\end{tabular}
\end{adjustbox}

Oba tematy można wykonać podobnie, obiekty samolotów czy statków składałyby się z kompozycji
elementów definiujących ich zachowanie - obiekt silnika informowałby o możliwościach
przemieszczania, pilot lub sonar pozwalałby na wyszukiwanie wrogich jednostek, obiekty
torped/karabinu przechowywałyby program unieszkodliwiania adwersarzy.

Wszystkie samoloty/łodzie podwodne rakiety/torpedy mogłyby dziedziczyć po klasie obiektu fizycznego
odpowiedzialnej za przemieszczanie zachowując pozory fizyki.

Utworzone obiekty powinny również implementować interfejs umożliwiający ich konwersję do
określonego formatu danych i z powrotem, w celu łatwego zapisywania i odczytywania stanu utworzonego
modelu.

\end{document}
