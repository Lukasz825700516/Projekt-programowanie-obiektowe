\documentclass{article}

\usepackage[utf8]{inputenc}
\usepackage{polski}
\usepackage[polish]{babel}

%% polish fancyref
\newcommand*{\fullref}[1]{\figurename~\ref{#1} na stronie~\pageref{#1}}

\usepackage[normalem]{ulem}
\usepackage{adjustbox}
\usepackage{array,booktabs}
\usepackage{amssymb}
\usepackage{listings}
\usepackage{xcolor}
\usepackage{tikz}
\usepackage{lscape}
\usepackage{calc}
\usepackage{lastpage}
\usepackage{fancyhdr}
\usepackage{rotating}

\pagestyle{fancy}
\fancyhf{}
\renewcommand{\headrulewidth}{0pt}
\fancyfoot[R]{Strona \thepage\ z \pageref{LastPage}}


\definecolor{codegreen}{rgb}{0,0.6,0}
\definecolor{codegray}{rgb}{0.5,0.5,0.5}
\definecolor{codepurple}{rgb}{0.58,0,0.82}
\definecolor{backcolour}{rgb}{0.95,0.95,0.92}

\lstdefinestyle{mystyle}{
    backgroundcolor=\color{backcolour},   
    commentstyle=\color{codegreen},
    keywordstyle=\color{magenta},
    numberstyle=\tiny\color{codegray},
    stringstyle=\color{codepurple},
    basicstyle=\ttfamily\footnotesize,
    breakatwhitespace=false,         
    breaklines=true,                 
    captionpos=b,                    
    keepspaces=true,                 
    numbers=left,                    
    numbersep=5pt,                  
    showspaces=false,                
    showstringspaces=false,
    showtabs=false,                  
    tabsize=2
}

\lstset{style=mystyle}


\makeatletter
\def\UL@putbox{\ifx\UL@start\@empty \else % not inner
  \vrule\@width\z@ \LA@penalty\@M
  {\UL@skip\wd\UL@box \UL@leaders \kern-\UL@skip}%
    \phantom{\box\UL@box}%
  \fi}
\makeatother

\begin{document}

\tableofcontents
\newpage

\section{Informacje}

Projekt z programowania obiektowego, symulacji agentowej
\underline{\textbf{bitwy samolotowej}}
przy użyciu języka programowania
\underline{\textbf{java}}.

\subsection{Grupa projektowa}
\begin{center}
	\begin{tabular}{cccc}
		Imię i Nazwisko & Rola & Numer indeksu & Termin labolatorii \\
		\hline Łukasz Mędrek & lider & 263910 & Środa 11:15 \\
		Jakub Kalecki & & 263962 & Środa 11:15 \\
	\end{tabular}
\end{center}

\subsection{Cel projektu}
Celem projektu jest zastosowanie technik projektowania obiektowego w  
w celu utworzenia modelu wybranego wycinka rzeczywistości i obserwacji
występujących w nim interakcji.

Projekt powinien mieć możliwość wprowadzenia danych początkowych, jak 
i zapisywania danych przeprowadzonej symulacji.

\subsection{Proponowane tematy}
\begin{adjustbox}{center}
	\begin{tabular}[p]{
			>{\centering\arraybackslash}c
			p{0.45\textwidth}
			p{0.45\textwidth}
			>{\centering\arraybackslash}c
		}
			Symulacja & \centering{Opis} & \centering{Cel} & Wybór \\
			% \hline Prosty ekosystem & {
				% Ekosystem składa się z różnych drapieżników, roślinożerców
				% i roślin, drapieżniki polują na roślinożerców, a roślinożercy
				% wędrują po planszy w poszukiwaniu roślin.
			% } & { 
				% Obserwacja cech każdej z grup w poszukiwaniu zależności
				% modyfikujących ich wielkość populacji.
			% }&  \\
			\hline \centering{Bitwa samolotowa} & { 
				Samoloty dwóch lub więcej stron prowadzą ze sobą
				bitwę w przestworzach, i mają za zadanie zestrzelić
				wrogie sobie jednostki.
			} & { 
				Obserwacja cech każdej z stron konfliktu w poszukiwaniu
				zależności modyfikujących ich szybkość eliminacji wrogich
				samolotów jak i końcową ilość jednostek.
			}& \checkmark \\
			\hline \centering{Statki} & { 
				Wersja gry w statki, w której każdy ze statków podwodnych może stwierdzić
				czy wokół niego znajduje się obca jednostka za pomocą radaru z daną dokładnością,
				wtedy rozpoczyna on ostrzał obszaru w którym wykrył wroga,
				próbując go wyeliminować. Statki mogą się poruszać, ale nie mogą się
				ze sobą zderzać.
			} & { 
				Obserwacja cech każdej z stron konfliktu w poszukiwaniu
				zależności modyfikujących ich szybkość eliminacji wrogich
				łodzi jak i końcową ilość jednostek.
			}&  \\
		\end{tabular}
\end{adjustbox}

Oba tematy można wykonać podobnie, obiekty samolotów czy statków składałyby się z kompozycji
elementów definiujących ich zachowanie - obiekt silnika informowałby o możliwościach
przemieszczania, pilot lub sonar pozwalałby na wyszukiwanie wrogich jednostek, obiekty
torped/karabinu przechowywałyby program unieszkodliwiania adwersarzy.

Wszystkie samoloty/łodzie podwodne rakiety/torpedy mogłyby dziedziczyć po klasie obiektu fizycznego
odpowiedzialnej za przemieszczanie zachowując pozory fizyki.

Utworzone obiekty powinny również implementować interfejs umożliwiający ich konwersję do
określonego formatu danych i z powrotem, w celu łatwego zapisywania i odczytywania stanu utworzonego
modelu.

\section{Projekt}
\subsection{Opis projektu}

Bitwa samolotowa jest symulacją mającą na celu sprawdzenie jak parametry samolotu wpływają
na jego siłę bojową, oraz całej floty.

\subsubsection{Agenci}

Głównymi jednostkami w symulacji są samoloty których struktura przedstawiona
jest na \fullref{pic:plane_class}, każdy z samolotów jest od siebie niezależny.

	\begin{sidewaysfigure}[p]
		\begin{adjustbox}{center}
			\resizebox{\minof{\pdfpageheight}{\height}}{!}{\input{uml/Plane.latex}}
		\end{adjustbox}
		\caption[] {
			Model klasy \lstinline{obiektowka.Plane} oraz klas bezpośrednio
			z nią związanych.
		}
		\label{pic:plane_class}
	\end{sidewaysfigure}

Wszystkie samoloty są umieszczone w symulacji \fullref{pic:simulation_class}, która
pozwala na ich wyszukiwanie, oraz łatwe uaktualnianie stanu.

	\begin{sidewaysfigure}[p]
		\begin{adjustbox}{center}
			\resizebox{\minof{\pdfpageheight}{\height}}{!}{\input{uml/Simulation.latex}}
		\end{adjustbox}
		\caption[Klasa \lstinline{obiektowka.Simulation}] {
			Model klasy \lstinline{obiektowka.Simulation} oraz klas bezpośrednio
			z nią związanych.
		}
		\label{pic:simulation_class}
	\end{sidewaysfigure}

Każdy agent ma swój stan, oraz przydzielone niezmienne parametry. Stanem
samolotów jest ich przyśpieszenie i pozycja w przestrzeni dwuwymiarowej, a reszta
informacji o nim jest podana jako referencja współdzielona przez inne jednostki.

Ta strategia ma na celu zminimalizować duplikację tych samych danych pomiędzy samolotami
- jest wiele różnych modeli samolotów, ale większość z nich różni się tylko kombinacją
użytych w nich części.

\subsubsection{Przebieg symulacji}

Symulacja odbywa się w czasie rzeczywistym, a każdy z symulowanych obiektów jest
odpowiedzialny za aktualizację swoich zmiennych.
Działania wykonywane przez samoloty są podejmowane przez pilotów \fullref{pic:full_pilot_interface}
znających dane swojego samolotu, i dane pobrane z otoczenia za pomocą przekazanego im obiektu
symulacji. Przykładowy stan obiektów podczas symulacji jest przedstawiony na \fullref{pic:object_diagram}

\subsubsection{Struktura projektu}

Aby umożliwić łatwy rozwój projektu po jego ukończeniu, został on rozdzielony na dwie części,
symulacyjną i wizualizacyjną. \lstinline{app} odpowiada za symulację, i zawiera on tylko kod 
za to odpowiedzialny. \lstinline{gui} odpowiada za wszelkie utworzone interfejsy wizualizacyjne.

\subsubsection{Wyświetlanie}

Wyświetlanie modelu powinno być od niego niezależne, dlatego w tym projekcie przedstawianie symulacji
odbywać się będzie za pomocą oddzielnej funkcji display \fullref{pic:display_class}, która będzie umożliwiała rysowanie danego
stanu symulacji za pomocą dowolnej implementacji. Funkcja ta miałaby tylko przyjmować model, pozycję
"kamery" oznaczającej widzialny obszar, który miałby być przekształcony na wyjściowy obraz, oraz rozmiar
widzialnej przestrzeni.
	\begin{sidewaysfigure}[p]
		\begin{adjustbox}{center}
			\resizebox{\minof{\pdfpageheight}{\height}}{!}{\input{uml/Gui.latex}}
		\end{adjustbox}
		\caption[Interfejs \lstinline{gui.Display}] {
			Interfejs \lstinline{gui.Display} określający cechy klas pozwalających na wyświetlanie symulacji
		}
		\label{pic:display_class}
	\end{sidewaysfigure}
	\begin{sidewaysfigure}[p]
		\begin{adjustbox}{center}
			\resizebox{\minof{\pdfpageheight}{\height}}{!}{\input{uml/object_diagram.latex}}
		\end{adjustbox}
		\caption[Diagram obiektów] {
			Przykładowy stan obiektów dla symulacji dwóch samolotów w różnych
			częściach symulacji, różniących się tylko pilotem.
		}
		\label{pic:object_diagram}
	\end{sidewaysfigure}
\end{document}
